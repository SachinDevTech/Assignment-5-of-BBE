\documentclass[12pts]{article}
\title{\textbf{NATIONAL INSTITUTE OF TECHNOLOGY, RAIPUR(C.G)}}
\usepackage{graphicx}
\graphicspath{{Images/}}


\author{SACHIN KUMAR\\sachin.1107sk@gmail.com\\Roll No: 21111047}
\date{February 23, 2022}
\begin{document}
\maketitle

\begin{figure}[h]
\centering
\includegraphics[scale=0.8]{nit.jpg}
\caption{National Institute of Technology, Raipur}
\end{figure}

\textbf{ASSIGNMENT-5 OF BASIC BIO-MEDICAL ENGINEERING}\\
\centering
UNDER THE SUPERVISION OF DR. SAURABH GUPTA SIR\\

\clearpage

\section*{Emerging Technologies in Healthcare}
\begin{figure}[h]
\centering
\includegraphics[scale=0.4]{emerge.jpg}
\caption{Emerging technologies in Healthcare}
\end{figure}

\subsection*{What is Healthcare Technology?}


\begin{figure}[h]
\centering
\includegraphics[scale=1]{tech.jpg}
\caption{Technology is relating us with healthcare}
\end{figure}


Healthcare technology refers to any IT tools or software designed to boost hospital and administrative productivity, give new insights into medicines and treatments, or improve the overall quality of care provided.\\
Currently we are looking for ways to improve in nearly every imaginable area. That’s where healthtech comes in. Tech-infused tools are being integrated into every step of our healthcare experience to counteract two key trouble spots: \begin{large}
\textit{quality and efficiency.}
\end{large}

\subsection*{How and where today's technology is helping us?}

It could facilitate tele-monitored surgeries, tele-education, tele-medicine and video consultations with doctors.
$Technology$ has also helped bring efficiencies in areas like CT (computed tomography) scans, such as better images and faster diagnosis.
Like these there are many reasons.\\

$Technology$ has made many aspects of life easier for people living in rural and remote areas. Applications that support instant communication and connection along with devices and machines that reduce the reliance on city or urban based facilities are some of the solutions that improve daily living. In healthcare, this has been a great challenge, especially for patients receiving ongoing or complex treatment for chronic issues, who may need to travel long distances for various healthcare services.

Although rural communities often have access to local healthcare facilities, there are many other factors that contribute to how they can access healthcare. A coordinated approach to healthcare that incorporates technology such as artificial intelligence is the ideal goal for rural communities.
By the help of technology we can get sustainability in the humans so that they can get quick access and quick responses,and quick diagnose.
It could help common men in rural areas,remote areas,hilly areas,sea areas,and such like many areas.





\subsection*{Artificial Intelligence}

\begin{figure}[h]
\centering
\includegraphics[scale=0.2]{AI.jpg}
\caption{Factors to conclude AI in healthcare}
\end{figure}

Artificial intelligence (AI), the ability of a digital computer or computer-controlled robot to perform tasks commonly associated with intelligent beings. The term is frequently applied to the project of developing systems endowed with the intellectual processes characteristic of humans, such as the ability to reason, discover meaning, generalize, or learn from past experience.

With the potential to radically transform healthcare, artificial intelligence can help professionals make better judgments and reduce human error and the risk of preventable scenarios. From radiology tools and immunotherapy for cancer patients to identifying infectious disease patterns, advanced technology helps develop more efficient and precise interventions.

\subsection*{Voice Search}

Voice search technology is becoming increasingly popular globally,many consumers prefer using voice-enabled search because it’s faster, easier, more convenient, and more natural. Most people would much rather speak than write. Additionally, when they ask Alexa or Siri a question, they can speak conversationally and provide more detail than they would if they were typing a query into their search browser window. 
One more example reveals that a mobile device’s ability to handle voice queries can help a patient who has a more detailed, more complex question. “Hey Siri or it may be 'Ok Google', find the best-rated pain doctor near me who offers nonsurgical pain relief.” For many people, talking simply comes more naturally than typing.
It can help us in the followings ways:
\begin{itemize}
\item Buying medications
\item Finding a physical therapist or physician
\item Scheduling an appointment
\item Using a tracker or health device
\end{itemize}



\subsection*{Virtual reality}
\begin{figure}[h]
\centering
\includegraphics[scale=0.2]{vr.jpg}
\caption{VR and AR image}
\end{figure}

Also an emerging technology in healthcare, virtual reality (VR) is an innovative tool with many applications, from teaching autistic children communication and social skills to engaging patients in activities and games for rehabilitation purposes. 
Virtual Reality has the ability to transport you inside the human body to access and view areas that otherwise would be impossible to reach. Currently, medical students learn on cadavers, which are difficult to get hold of and (obviously) do not react in the same way a live patient would. In VR however, you can view minute detail of any part of the body in stunning 360° Computer generated imagery (CGI) reconstruction  create training scenarios which replicate common surgical procedures.


\subsubsection*{Augmented reality}
Augmented reality (AR) is the art of superimposing computer-generated content over a live view of the world. AR integrates digital information with the user’s environment in real time and is becoming more accessible and affordable for medical education and imaging, dentistry, and nurse training.
AR enhances visualization of CT or MRI data by superimposing stereoscopic projections during a surgical procedure.

\subsection*{Mobile Apps}


\begin{figure}[h]
\centering
\includegraphics[scale=0.18]{mobile.jpg}
\caption{By mobile apps you can get real time health data}
\end{figure}

Relatively new technology in healthcare, mobile applications can upload patients’ medical records, check in, schedule appointments, and provide expert advice. A host of different mobile apps now helps physicians with patient monitoring and management, information gathering, consulting, and health record access and maintenance. 

Applications also assist healthcare professionals with medical training and education, clinical decision-making, and time and data management. They come in handy in providing physicians with communication tools such as email, text, video conferencing, and voice calling.Physicians can also access the devices from wherever they are treating patient.Thus it is helpful.


\subsection*{Telemedicine}

\begin{figure}[h]
\centering
\includegraphics[scale=0.2]{tel.jpg}
\caption{Telemedicine visual Graph}
\end{figure}

The purpose of telemedicine is to improve a patient’s health by enabling two-way, real-time communication between a patient and a healthcare practitioner at a distant site. Initially, the aim of creating telemedicine providers was to treat and cater to the patients living in remote areas where there was a shortage of medical facilities. While it is still being used to address these problems, it’s now also becoming a tool of convenience for the patients.
Through the platform, a patient can discuss his/her symptoms, medical issues, and more with a practitioner in real-time. The patients will be able to receive diagnoses, learn their treatment options, and get a prescription while being in the comfort of their homes.
it's pros may be like as convenient and easily accessible, Saving on healthcare expenditure, Personalized Patient Engagement etc.




\end{document}